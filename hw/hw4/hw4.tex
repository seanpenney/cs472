\documentclass[letterpaper,10pt,titlepage]{article}

\usepackage{graphicx}                                        
\usepackage{amssymb}                                         
\usepackage{amsmath}                                         
\usepackage{amsthm}                                          

\usepackage{alltt}                                           
\usepackage{float}
\usepackage{color}
\usepackage{url}

\usepackage{balance}
\usepackage[TABBOTCAP, tight]{subfigure}
\usepackage{enumitem}
\usepackage{pstricks, pst-node}

\usepackage{geometry}
\geometry{textheight=9in, textwidth=6.5in}

\newcommand{\cred}[1]{{\color{red}#1}}
\newcommand{\cblue}[1]{{\color{blue}#1}}

\usepackage{hyperref}
\usepackage{geometry}

\usepackage{listings}


\def\name{Sean Penney and Paul Atkinson}

%% The following metadata will show up in the PDF properties
\hypersetup{
  colorlinks = true,
  urlcolor = black,
  pdfauthor = {\name},
  pdfkeywords = {cs472 ``computer architecture'' clements ``chapter 3''},
  pdftitle = {CS 472: Homework 4},
  pdfsubject = {CS 472: Homework 4},
  pdfpagemode = UseNone
}

\begin{document}
\hfill \name

\hfill \today

\hfill CS 472 HW 4

\begin{enumerate}
\item[$(3.9)$] 

  A misaligned operand is one that is not aligned on the word boundaries of the system.
  
  Misaligned operands are an issue, because 2 memory location may need to be read to get an operand if it is not properly aligned.
  Reading 2 memory locations to get one operand is inefficient

\item[$(3.26)$] 

  The register r6 shifted to the left by the value in r2.  The result is added to r5.  This becomes the effective address for data that is loaded into r0.
  
\item[$(3.33)$] 

  Block move instructions are useful since a block of registers can be copied to or from memory with a single instruction.
  Block moves are easy to understand, however, there are many options that determine how the move takes place which can complicate things.

\item[$(3.36)$] 

  Let r0 be the register we are multiplying by a given value.
  
  a. (r0 << 5) + r0
  
  b. (ro << 10) + r0
  
  c. (r0 << 12) - r0 

\item[$(3.48)$]

  A pseudo-operation do not directly translate to a machine instruction.  They are resolved by the assembler during assembly.
  Pseudo operations generally give information such as data alignment or symbol definitions.
  
\item[$(3.60)$]

\begin{lstlisting}
AREA vector, CODE, READONLY
		ENTRY

VECTOR	MOV r0, #8			; loop 8 times
	ADR r1, VECA
	ADR r2, VECB
	ADR r3, VECC
LOOP	LDR r4, [r1], #4	; get element from VECA, post increment address
	LDR r5, [r2], #4	; get element from VECB, post increment address
	ADD r6, r4, r5		; add elements from VECA and VECB
	LSR r6, r6, #1		; shift result to right (divide by 2)
	STR r6, [r3], #4	; store result in VECC and post increment address
	SUBS r0, r0, #1		; decrement loop counter and set status flag
	BNE LOOP			; continue untsil loop counter is 0
	MOV pc, lr			; Return from subroutine
		
AREA vector, DATA, READWRITE
VECA	DCD 1,2,3,4,5,6,7,8 ; 8 element vector
VECB	DCD 1,2,3,4,5,6,7,8 ; 8 element vector
VECC	DCD 0,0,0,0,0,0,0,0 ; 8 element vector
\end{lstlisting}
  
\item[$(3.61)$]

  The register r15 could not be used in conjunction with most data processing instructions because the program counter has certain values that would be valid.
  Instructions are four bytes long, so r15 can only be changed in increments of 4.  Also, the PC has to point to an actual instruction in the program.
  
\end{enumerate}



\end{document}
