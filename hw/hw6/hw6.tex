\documentclass[letterpaper,10pt,titlepage]{article}

\usepackage{graphicx}                                        
\usepackage{amssymb}                                         
\usepackage{amsmath}                                         
\usepackage{amsthm}                                          

\usepackage{alltt}                                           
\usepackage{float}
\usepackage{color}
\usepackage{url}

\usepackage{balance}
\usepackage[TABBOTCAP, tight]{subfigure}
\usepackage{enumitem}
\usepackage{pstricks, pst-node}

\usepackage{geometry}
\geometry{textheight=9in, textwidth=6.5in}

\newcommand{\cred}[1]{{\color{red}#1}}
\newcommand{\cblue}[1]{{\color{blue}#1}}

\usepackage{hyperref}
\usepackage{geometry}

\usepackage{listings}


\def\name{Sean Penney and Paul Atkinson}

%% The following metadata will show up in the PDF properties
\hypersetup{
  colorlinks = true,
  urlcolor = black,
  pdfauthor = {\name},
  pdfkeywords = {cs472 ``computer architecture'' clements ``chapter 6''},
  pdftitle = {CS 472: Homework 6},
  pdfsubject = {CS 472: Homework 6},
  pdfpagemode = UseNone
}

\begin{document}
\hfill \name

\hfill \today

\hfill CS 472 HW 6

\begin{enumerate}
\item[$(6.1)$] 

  There are many criteria for judging computer performance.
  These include efficiency, throughput, latency, relative performance, and time and rate.
  Performance is difficult to define because of the different criteria, and also because different people are concerned with different performance criteria.

\item[$(6.4)$]

  Efficiency = bits of data words / total bits sent = 16384 / 16608 = 98.65%
  
\end{enumerate}

\end{document}
