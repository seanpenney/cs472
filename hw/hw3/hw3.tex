\documentclass[letterpaper,10pt,titlepage]{article}

\usepackage{graphicx}                                        
\usepackage{amssymb}                                         
\usepackage{amsmath}                                         
\usepackage{amsthm}                                          

\usepackage{alltt}                                           
\usepackage{float}
\usepackage{color}
\usepackage{url}

\usepackage{balance}
\usepackage[TABBOTCAP, tight]{subfigure}
\usepackage{enumitem}
\usepackage{pstricks, pst-node}

\usepackage{geometry}
\geometry{textheight=9in, textwidth=6.5in}

\newcommand{\cred}[1]{{\color{red}#1}}
\newcommand{\cblue}[1]{{\color{blue}#1}}

\usepackage{hyperref}
\usepackage{geometry}

\usepackage{listings}


\def\name{Sean Penney and Paul Atkinson}

%% The following metadata will show up in the PDF properties
\hypersetup{
  colorlinks = true,
  urlcolor = black,
  pdfauthor = {\name},
  pdfkeywords = {cs472 ``computer architecture'' clements ``chapter 3''},
  pdftitle = {CS 472: Homework 3},
  pdfsubject = {CS 472: Homework 3},
  pdfpagemode = UseNone
}

\begin{document}
\hfill \name

\hfill \today

\hfill CS 472 HW 3

\begin{enumerate}
\item[$(3.1)$] 

  The program counter is not a counter, it holds, or points to, the memory address of the next instruction to be executed.

\item[$(3.3)$]

  a.  PC points to the memory address of the next instruction to be executed.
  
  b.  MAR stores the address that is being accessed by read/write operations.
  
  c.  MBR is the memory buffer register, which holds data that has been read from main memory or will be written to main memory.
  
  d.  The IR (instruction register) holds the instruction currently being executed.
  
\item[$(3.17)$]

  

\item[$(3.19)$]   
  
  Using the XOR swap algorithm:
  
  EOR r0, r0, r1
  EOR r1, r1, r0
  EOR r0, r0, r1
  
\item[$(3.39)$]

LOOP	LDRB r1, [r0], \#1	; get first character
		CMP r1
  
\end{enumerate}



\end{document}
