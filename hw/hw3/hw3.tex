\documentclass[letterpaper,10pt,titlepage]{article}

\usepackage{graphicx}                                        
\usepackage{amssymb}                                         
\usepackage{amsmath}                                         
\usepackage{amsthm}                                          

\usepackage{alltt}                                           
\usepackage{float}
\usepackage{color}
\usepackage{url}

\usepackage{balance}
\usepackage[TABBOTCAP, tight]{subfigure}
\usepackage{enumitem}
\usepackage{pstricks, pst-node}

\usepackage{geometry}
\geometry{textheight=9in, textwidth=6.5in}

\newcommand{\cred}[1]{{\color{red}#1}}
\newcommand{\cblue}[1]{{\color{blue}#1}}

\usepackage{hyperref}
\usepackage{geometry}

\usepackage{listings}


\def\name{Sean Penney and Paul Atkinson}

%% The following metadata will show up in the PDF properties
\hypersetup{
  colorlinks = true,
  urlcolor = black,
  pdfauthor = {\name},
  pdfkeywords = {cs472 ``computer architecture'' clements ``chapter 3''},
  pdftitle = {CS 472: Homework 3},
  pdfsubject = {CS 472: Homework 3},
  pdfpagemode = UseNone
}

\begin{document}
\hfill \name

\hfill \today

\hfill CS 472 HW 3

\begin{enumerate}
\item[$(3.1)$] 

  The program counter is not a counter.  The PC holds, or points to, the memory address of the next instruction to be executed.

\item[$(3.2)$]

  a.  PC points to the memory address of the next instruction to be executed.
  
  b.  MAR stores the address that is being accessed by read/write operations.
  
  c.  MBR is the memory buffer register, which holds data that has been read from main memory or will be written to main memory.
  
  d.  The IR (instruction register) holds the instruction currently being executed.

\item[$(3.3)$]

  a.  C = 0, Z = 0, V = 0, N = 0
  
  b.  C = 1, Z = 1, V = 0, N = 0
  
  c.  C = 0, Z = 0, V = 0, N = 0
  
  d.  C = 1, Z = 0, V = 0, N = 0
  
  e.  C = 0, Z = 0, V = 0, N = 1
  
  f.  C = 1, Z = 0, V = 0, N = 1
  
\item[$(3.10)$]
  
  RSB (Reverse Subtract) exists because there are a wide range of options for Operand2. Operand2 can be either a constant or a 
  register with optional shift.
  
\item[$(3.17)$]

  Using an 8-bit format for the integer and a 4-bit alignment field allows for a larger range of values.
  The disadvantage to this mechanism is that there are gaps in the range of values.

\item[$(3.18)$]
  
  AND r0, 0xFE0FFFFF
  
  This will take all the bits from 20-25 and set them to zero and it will leave the rest alone.
  
\item[$(3.19)$]   
  
  Using the XOR swap algorithm:
  
  EOR r0, r0, r1
  
  EOR r1, r1, r0
  
  EOR r0, r0, r1

\item[$(3.25)$]
  
  a. 11100101110000100001000000000000
  
  b. 11100111101101000001000000000101
  
  c. 11100110100101000011000000000101
  
  d. 11100101001101000011000000000110
  
\item[$(3.39)$]

LOOP	LDRB r2, [r0], \#1	; get address of next character

		STRB, r2, [r1], \#1	; store contents at r2 in r1

		TEQ r2, \#0			; check if at end of string

		BRNE LOOP			; if not at end of string, go back to loop
		
  
\item[$(3.51)$]
  
\begin{lstlisting}

		AREA palindrome, CODE, READONLY
		ENTRY
							
start	LDR r1, =string_begin		
		LDR r2, =string_end
		BL pal				
stop	B stop

pal		LDRB r3, [r1], #1	; get left hand character
		LDRB r4, [r2], #-1	; get right hand character
		CMP r3, r4			; compare the ends of the string
		BNE notpal			; if different then fail
		SUBS r3, r2, r1		; get difference between pointers
		BEQ waspal			; if same then exit with palindrome found
		BMI waspal			; if left pointer past right then palindrome
		B pal				; REPEAT
waspal	MOV r0, #0x1		; r0 = 1 = success flag
		MOV pc, lr			; return
notpal	MOV r0, #0x0		; r0 = 0 = fail flag
		MOV pc, lr			; return
		
		END

\end{lstlisting}
  
\end{enumerate}



\end{document}
