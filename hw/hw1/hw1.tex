\documentclass[letterpaper,10pt,titlepage]{article}

\usepackage{graphicx}                                        
\usepackage{amssymb}                                         
\usepackage{amsmath}                                         
\usepackage{amsthm}                                          

\usepackage{alltt}                                           
\usepackage{float}
\usepackage{color}
\usepackage{url}

\usepackage{balance}
\usepackage[TABBOTCAP, tight]{subfigure}
\usepackage{enumitem}
\usepackage{pstricks, pst-node}

\usepackage{geometry}
\geometry{textheight=9in, textwidth=6.5in}

\newcommand{\cred}[1]{{\color{red}#1}}
\newcommand{\cblue}[1]{{\color{blue}#1}}

\usepackage{hyperref}
\usepackage{geometry}

\usepackage{listings}

\def\name{Sean Penney and Paul Atkinson}

%% The following metadata will show up in the PDF properties
\hypersetup{
  colorlinks = true,
  urlcolor = black,
  pdfauthor = {\name},
  pdfkeywords = {cs472 ``computer architecture'' clements ``chapter 1''},
  pdftitle = {CS 472: Homework 1},
  pdfsubject = {CS 472: Homework 1},
  pdfpagemode = UseNone
}

\begin{document}
\hfill \name

\hfill \today

\hfill CS 472 HW 1

\begin{enumerate}
\item[$(1.3)$] We said that the pattern of $1$s and $0$s used to represent an instruction
  in a computer has no intrinsic meaning. Why is this so and what is the implication of
  this statement?

  The pattern of $1$s and $0$s has no intrinsic meaning because this allows the $1$s and $0$s
  to represent different things.  Assembly instructions can be translated into $1$s and $0$s,
  and then the hardware knows what to execute based on the pattern of $1$s and $0$s.
  
  %Answer goes here -- make sure you put a blank line between question and answer.
\item[$(1.5)$] Modify the algorithm used in this chapter to locate the longest run of
  non-consecutive characters in the string.
  
  The following is the algorithm from page 31 of the textbook, with a modification so that
  it finds the longest run of non-consecutive characters instead of consecutive.
  
\end{enumerate}
\begin{lstlisting}
Read the first digit in the string and call it New_Digit
Set the Current_Run_Value to New_Digit
Set the Current_Run_Length to 1
Set the Max_Run to 1
REPEAT
Read the next digit in the sequence (i.e., read New_Digit)
IF its value is not the same as Current_Run_Value
	THEN Current_Run_Length = Current_Run_Length + 1
	ELSE {Current_Run_Length = 1
		Current_Run_Value = New_Digit}
IF Current_Run_Length > Max_Run
	THEN Max_Run = Current_Run_Length
UNTIL The last digit is read

\end{lstlisting}
\begin{enumerate}
\item[$(1.8)$] What are the differences between RTL, machine language, assembly language,
  high-level language, and pseudocode?
  
  Assembly language is a low level programming language that is specific to an architecture.
  In contrast, register transfer language (RTL) is generalized and not specific to a hardware implementation.
  An assembler takes the assembly code, and converts the code to machine language.  Machine language can be
  directly executed by the computer hardware.
  A high-level language has broad functionality and is easier to program compared to assembly language.
  However, programming in a high-level language will not be as efficient since it needs to be translated to 
  assembly language, and the translation might not be performed in the most efficient way.
  Pseudocode describes how an algorithm would be implemented as a program, without using specific syntax.
  
  
\item[$(1.12)$]What is the difference between a computer's \textit{architecture} and its
  \textit{organization}?
  
  A computer's architecture relates to the instruction set, and the organization is the actual hardware configuration of the machine.
  
\item[$(1.18)$]What is the von Neumann bottleneck?

  The von Neumann bottleneck is a limiting factor in the efficiency of data transfer between the CPU and memory.
  Program memory and data memory cannot be accessed at the same time, which leads to the CPU waiting for data from the memory at times.

\item[$(1.33)$]Is Moore's law a law?

  Moore's law is not a law, it is an observation that the number of components in an integrated circuit doubles every two years.

\end{enumerate}



\end{document}
