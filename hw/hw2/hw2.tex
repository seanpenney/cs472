\documentclass[letterpaper,10pt,titlepage]{article}

\usepackage{graphicx}                                        
\usepackage{amssymb}                                         
\usepackage{amsmath}                                         
\usepackage{amsthm}                                          

\usepackage{alltt}                                           
\usepackage{float}
\usepackage{color}
\usepackage{url}

\usepackage{balance}
\usepackage[TABBOTCAP, tight]{subfigure}
\usepackage{enumitem}
\usepackage{pstricks, pst-node}

\usepackage{geometry}
\geometry{textheight=9in, textwidth=6.5in}

\newcommand{\cred}[1]{{\color{red}#1}}
\newcommand{\cblue}[1]{{\color{blue}#1}}

\usepackage{hyperref}
\usepackage{geometry}

\usepackage{listings}


\def\name{Sean Penney and Paul Atkinson}

%% The following metadata will show up in the PDF properties
\hypersetup{
  colorlinks = true,
  urlcolor = black,
  pdfauthor = {\name},
  pdfkeywords = {cs472 ``computer architecture'' clements ``chapter 2''},
  pdftitle = {CS 472: Homework 2},
  pdfsubject = {CS 472: Homework 2},
  pdfpagemode = UseNone
}

\begin{document}
\hfill \name

\hfill \today

\hfill CS 472 HW 2

\begin{enumerate}
\item[$(2.5)$] 

  5 decimal digits are necessary to represent the number to the given precision.
  
  To find the number of bits to represent the number:
  
  $log_{2}(10^5) = 17$, so 17 bits are required.

\item[$(2.13)$]  

  a.  0b10010010
  
  b.  0b10001000
  
  c.  0x0A135152
  
  d.  0x0FAC4D2E
  
 \item[$(2.14)$]
  Arithmetic overflow occurs when adding two values that gives an answer that is bigger then the amount of bits available
  (such as adding two eight bit numbers that add up to 280). Arithmetic overflow can be detected by checking if adding two positive numbers
  results in a negative number or adding two negative numbers results in a positive.
  
\item[$(2.16)$]

  Sign bit: 0
  Exponent: 10001001
  Significand: 00110100100010000000000
  
  $1234.125 = 2^{10} * 1.205$....
  
\item[$(2.17)$]

  Sign bit: 0
  Exponent: 10011110
  Significand: 10011000100110000000000
  
  $3427532800 = 2^{31} * 1.5960693359375$
  
\item[$(2.22)$]

  Truncation error occurs when infinite sums have to be approximated by a finite sum.
  
  Rounding error occurs when a finite number has to be rounded to a number that can be represented within the limits of the hardware.
  
\item[$(2.40)$]

\begin{tabular}{c|c|c}
A & B & Answer \\
\hline
0 & 0 & 0 \\
0 & 1 & 1 \\
1 & 0 & 1 \\
1 & 1 & 0 \\
\end{tabular}

The output is true when A and B are different, so it is a XOR gate.

\item[$(2.45)$]

 
  AND table:
\begin{tabular}{c|c|c|c}
A & B & C & Answer \\
\hline
0 & 0 & 0 & 0\\
0 & 0 & 1 & 0\\
0 & 1 & 0 & 0\\
0 & 1 & 1 & 0\\
1 & 0 & 0 & 0\\
1 & 0 & 1 & 0\\
1 & 1 & 0 & 0\\
1 & 1 & 1 & 1\\
\end{tabular}

  OR table:
\begin{tabular}{c|c|c|c}
A & B & C & Answer \\
\hline
0 & 0 & 0 & 0\\
0 & 0 & 1 & 1\\
0 & 1 & 0 & 1\\
0 & 1 & 1 & 1\\
1 & 0 & 0 & 1\\
1 & 0 & 1 & 1\\
1 & 1 & 0 & 1\\
1 & 1 & 1 & 1\\
\end{tabular}

  NAND table:
\begin{tabular}{c|c|c|c}
A & B & C & Answer \\
\hline
0 & 0 & 0 & 1\\
0 & 0 & 1 & 1\\
0 & 1 & 0 & 1\\
0 & 1 & 1 & 1\\
1 & 0 & 0 & 1\\
1 & 0 & 1 & 1\\
1 & 1 & 0 & 1\\
1 & 1 & 1 & 0\\
\end{tabular}

  NOR table:
\begin{tabular}{c|c|c|c}
A & B & C & Answer \\
\hline
0 & 0 & 0 & 1\\
0 & 0 & 1 & 0\\
0 & 1 & 0 & 0\\
0 & 1 & 1 & 0\\
1 & 0 & 0 & 0\\
1 & 0 & 1 & 0\\
1 & 1 & 0 & 0\\
1 & 1 & 1 & 0\\
\end{tabular}

  It is not possible to have an n input XOR gate for n$>$2.
  An XOR gate evaluates true if the inputs are opposite.  This works fine if there are two inputs.
  With 3 inputs, however, we could run into issues if the first two inputs are true and the third is false, for example.
  Given the first two inputs, it would evaluate to false.  Since the first and third inputs would be different, it would also evaluate to true.
  
  XOR table:
\begin{tabular}{c|c|c|c}
A & B & C & Answer \\
\hline
0 & 0 & 0 & 1\\
0 & 0 & 1 & Not defined\\
0 & 1 & 0 & Not defined\\
0 & 1 & 1 & Not defined\\
1 & 0 & 0 & Not defined\\
1 & 0 & 1 & Not defined\\
1 & 1 & 0 & Not defined\\
1 & 1 & 1 & 0\\
\end{tabular}
  
\end{enumerate}



\end{document}
